\documentclass[10pt]{article}
\input{style/coursHeadings}
\input{style/programHeadings}
\input{style/macros_SII}
\input{style/macros_Titres}
\input{style/macros_Frames}

%Si le boolen xp est vrai : compilation pour xabi
%Sinon compilation Damien
\newboolean{xp}
\setboolean{xp}{true}

\newboolean{prof}
\setboolean{prof}{true}

\usepackage[%
    pdftitle={CI5 SN - Systèmes Séquentiels},
    pdfauthor={Xavier Pessoles},
    colorlinks=true,
    linkcolor=blue,
    citecolor=magenta]{hyperref}


\def\discipline{Sciences Industrielles de l'Ingénieur}
\def\xxtitre{\ifthenelse{\boolean{xp}}{
CI 5 : Étude du comportement des systèmes numériques}{
Chapitre  -- }}

\def\xxsoustitre{\ifthenelse{\boolean{xp}}{
Chapitre 2 -- Étude des systèmes séquentiels}{
Partie  -- }}

\def\xxauteur{\ifthenelse{\boolean{xp}}{
Xavier \textsc{Pessoles} \\ 2013 -- 2014}{
}}

\def\xxpied{\ifthenelse{\boolean{xp}}{
CI 5 : Étude du comportement des systèmes numériques\\
Chapitre 2 -- Étude des systèmes séquentiels -- TD}{
\xxtitre}}

\def\xxcathegorie{\ifthenelse{\boolean{xp}}{
2013 -- 2014 \\
Xavier \textsc{Pessoles}}{}}





%---------------------------------------------------------------------------


\begin{document}

\ifthenelse{\boolean{xp}}{\input{style/enteteXP}}{\input{style/enteteDI}}


\begin{flushright}
\textit{D'après ressources de David Prévost.}
\end{flushright}

\section{Doseur pondéral DPX}
\begin{center}
\includegraphics[width=.5\textwidth]{images/fig_01}
\end{center}
On s'intéresse à un système de production de pièces en matières plastiques. La presse à injecter
n’est pas l’objet de l’étude mais les fonctions à réaliser par cette presse sont évidemment liées
aux fonctions du doseur pondéral qui est l’objet principal de cet exercice.
La matière plastique « première » sous forme de différents granulés (produits principaux :
macro produits et colorants : micro produits) est conservée dans des silos de grande taille. Le
transport de ces matières vers les presses est réalisé automatiquement au moyen de
canalisations et d’aspirateurs dans des trémies de stockage situées au-dessus d’un doseur
pondéral qui permet le dosage et le mélange des granulés.

Le doseur pondéral DPX produit un mélange dosé de matières premières par lot. Une quantité
unitaire de mélange dosé s’appelle un « batch ». A l’ouverture de la trémie de pesée, un batch
tombe dans la zone de malaxage du mélange. Cette zone de mélange alimente directement par
gravité la presse à injecter (doseur au-dessus de l’entrée de la presse à injecter. Le procédé vise
à obtenir un dosage précis des macro produits et micro produits en faisant en sorte que les
masses de ces produits dans un batch soient conformes à la consigne.

\begin{center}
\includegraphics[width=.6\textwidth]{images/fig_02}
\end{center}

Le doseur pondéral permet d’alimenter la trémie de pesée :
\begin{itemize}
\item en macro produit stocké dans la partie supérieure au moyen d’un godet de vidage;
\item en micro produit stocké latéralement au moyen d’une vis d’Archimède.
\end{itemize}

La dose totale s’appelle un « batch ». Une fois le batch pesé il est vidé dans le mélangeur à hélice
qui permet d’homogénéiser le produit qui chute par gravité dans le dispositif d’alimentation de
la presse à injecter. Le capteur de niveau surveille la consommation de mélange par la presse et
arrête le fonctionnement du DPX en cas de niveau trop élevé

Le fonctionnement du doseur peut-être décrit par l’algorigramme suivant :

\begin{center}
\includegraphics[width=.75\textwidth]{images/fig_03}
\end{center}

\begin{hypo}
On supposera dans la suite que le mélangeur n’est pas plein (capteur de
niveau de mélange non activé).
\end{hypo}

\subparagraph{}
\textit{Compléter le chronogramme donné sur le document réponse et en déduire le temps de
cycle de réalisation d’un mélange avec le doseur pondéral DPX.}


\begin{center}
\includegraphics[width=.95\textwidth]{images/fig_04}
\end{center}

\section*{Panne d'un hydro planeur}
\setcounter{subparagraph}{0}

\begin{center}
\includegraphics[width=.5\textwidth]{images/fig_05}
\end{center}
Dans l'objectif d'optimiser le fonctionnement d'un hydro planeur il faut tenir compte de toutes
les procédures de fonctionnement prévues, comme celle d'alerte en cas de panne de la
transmission des données, qui impose d'émettre un signal de détresse permettant de venir
repêcher l'hydro planeur.

Dans ce cas de dysfonctionnement, l'hydro planeur adopte le comportement décrit par le
diagramme d'état ci-dessous :

\begin{center}
\includegraphics[width=.85\textwidth]{images/fig_06}
\end{center}

\subparagraph{}
\textit{Compléter les chronogrammes du document qui correspondent à la séquence des
signaux de commande fournis par l’unité de traitement pour obtenir le fonctionnement
souhaité dans le cas où la première et la deuxième transmission IRIDIUM échouent
(lorsqu’un élément doit être activé, il sera représenté par un niveau haut).}



\begin{center}
\includegraphics[width=.75\textwidth]{images/fig_07}
\end{center}

\section*{Soufflerie équipant un tunnel souterrain}

\begin{center}
\includegraphics[width=.95\textwidth]{images/fig_08}
\end{center}

\subsection*{Présentation}
Afin de faciliter les déplacements des marchandises et des personnes sur Terre, l’usage de
tunnels souterrains est de plus en plus fréquent. Ces tunnels doivent assurer la sécurité des
personnes les empruntant, surtout s’il s’agit de tunnels autoroutiers où des véhicules
consommant du carburant fossile les empruntent. Ces véhicules majoritairement constitués
d’un moteur thermique consomment du pétrole (gazole, essence, …) et dégagent des gaz,
notamment des oxydes d’Azote, du dioxyde de Soufre, du gaz $CO_2$, ainsi que bon nombre de
particules dangereuses telles que le Plomb, …

Afin d’assurer la sécurité des personnes dans les tunnels et de respecter la législation en vigueur
sur ces ouvrages, il est nécessaire d’installer un système de ventilation permettant la circulation
de l’air, et donc d’évacuer les gaz.

On envisage dans cette étude de proposer un modèle de comportement de la commande de cette
ventilation décrit par l’outil graphe d’états (diagramme état-transition). Un tunnel est équipé de
3 ventilateurs indépendants et de 2 capteurs (un de température et un de gaz $CO_2$).

La commande de chaque ventilateur (variables de sortie) est indépendante et sont notées
$Fan1$, $Fan2$ et $Fan3$ et sont des variables binaires.

Les 2 capteurs permettent d’acquérir l’état de l’air dans le tunnel :
\begin{itemize}
\item la variable associée au capteur de température est $temp$;
\item la variable associée au capteur de gaz $CO_2$ est notée $CO2$.
\end{itemize}

Dans un premier temps, on souhaite décrire le comportement de la ventilation en accord avec le cahier
des charges fonctionnel présenté ci-dessous.

Le premier ventilateur $Fan1$ est toujours commandé afin de créer un léger flux d’air dans le
tunnel. Le second ventilateur $Fan2$ est commandé (ventilateur $Fan3$ non commandé) lorsque la
température dans le tunnel dépasse 20\textdegree C et qu’il n’y a pas de $CO_2$. Le troisième ventilateur $Fan3$
est commandé (avec le ventilateur $Fan2$) lorsque que le capteur de gaz $CO_2$ indique une
présence importante de gaz $CO_2$. Si la température est supérieure à 24\textdegree C et que le niveau de gaz
$CO_2$ n’est pas trop important, la commande des ventilateurs 1 et 3 est réalisée (ventilateur Fan2
non commandé).

\subsection*{Travail demandé}
Une étude préliminaire a permis de mettre en évidence la présence de 4 états sur le système de
ventilation.

\setcounter{subparagraph}{0}
\subparagraph{}
\textit{Listez ces 4 états en précisant pour chacun d’entre eux le ou les ventilateurs commandés.}

On donne ci-dessous le graphe d’état permettant de décrire le fonctionnement séquentiel
souhaité :


\begin{center}
\includegraphics[width=.5\textwidth]{images/fig_09}
\end{center}

L’objectif est dans un premier temps d’analyser ce graphe d’état, puis dans un second temps, de
proposer une amélioration suite à une évolution du cahier des charges.

\subparagraph{}
\textit{Analyser les évolutions possibles du graphe d’état lors de la mise sous tension. Justifier la
présence de la transition entre l’Etat1 (état source) et l’Etat3 (état destination).}

\subparagraph{}
\textit{En fonctionnement normal (en dehors de la mise sous tension), justifier le fait que la
séquence Etat1 $\rightarrow$ Etat2 $\rightarrow$ Etat1 $\rightarrow$ Etat3 est impossible. Indiquer précisément la raison.}

\subparagraph{}
\textit{Compléter ci-dessous, le chronogramme en spécifiant l’état actif du diagramme état transition et l’état des ventilateurs. \`A l'instant $t=0$, l’état actif est l’Etat1.}

\begin{center}
\includegraphics[width=.95\textwidth]{images/fig_10}
\end{center}

Dans un second temps, on souhaite faire évoluer le cahier des charges fonctionnel, tel que décrit
ci-dessous.

Extrait du nouveau cahier des charges fonctionnel :
Le nouveau cahier des charges est en tout point identique au précédent, mais prend en compte
la commande des 3 ventilateurs en cas d’incendie (variable binaire Feu).

\subparagraph{}
\textit{Décrire sur le graphe d’état précédent, et en rouge, le comportement de la ventilation
respectant le nouveau cahier des charges. Vous prendrez soin de vérifier les propriétés de
complétude et de non contradiction.}


\end{document}


