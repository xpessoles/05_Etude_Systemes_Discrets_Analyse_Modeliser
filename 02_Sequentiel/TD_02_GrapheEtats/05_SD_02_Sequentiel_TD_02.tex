\documentclass[10pt]{article}
\input{style/coursHeadings}
\input{style/programHeadings}
\input{style/macros_SII}
\input{style/macros_Titres}
\input{style/macros_Frames}

%Si le boolen xp est vrai : compilation pour xabi
%Sinon compilation Damien
\newboolean{xp}
\setboolean{xp}{true}

\newboolean{prof}
\setboolean{prof}{true}

\usepackage[%
    pdftitle={CI5 SN - Systèmes Séquentiels},
    pdfauthor={Xavier Pessoles},
    colorlinks=true,
    linkcolor=blue,
    citecolor=magenta]{hyperref}


\def\discipline{Sciences Industrielles de l'Ingénieur}
\def\xxtitre{\ifthenelse{\boolean{xp}}{
CI 5 : Étude du comportement des systèmes numériques}{
Chapitre  -- }}

\def\xxsoustitre{\ifthenelse{\boolean{xp}}{
Chapitre 2 -- Étude des systèmes séquentiels}{
Partie  -- }}

\def\xxauteur{\ifthenelse{\boolean{xp}}{
Xavier \textsc{Pessoles} \\ 2013 -- 2014}{
}}

\def\xxpied{\ifthenelse{\boolean{xp}}{
CI 5 : Étude du comportement des systèmes numériques\\
Chapitre 2 -- Étude des systèmes séquentiels -- TD}{
\xxtitre}}

\def\xxcathegorie{\ifthenelse{\boolean{xp}}{
2013 -- 2014 \\
Xavier \textsc{Pessoles}}{}}





%---------------------------------------------------------------------------


\begin{document}

\ifthenelse{\boolean{xp}}{\input{style/enteteXP}}{\input{style/enteteDI}}


\begin{flushright}
\textit{D'après ressources de David Prévost.}
\end{flushright}

\section{Poste de dégraissage d’une machine de traitement de surface}
\begin{center}
\includegraphics[width=.5\textwidth]{images/fig_01}
\end{center}
On s'intéresse à un poste de dégraissage d’une machine de dégraissage dont on donne une
description matérielle ainsi qu’un extrait de cahier des charges ci-dessous.

Le poste de dégraissage est utilisé pour décaper des pièces avant un traitement de surface. Il se
compose d’une zone de chargement, d’une zone de déchargement, d’une cuve de dégraissage et
d’un chariot automoteur se déplaçant sur un rail. Ce chariot permet de déplacer un panier
contenant les pièces à traiter.

\begin{center}
\includegraphics[width=.5\textwidth]{images/fig_02}
\end{center}

Extrait du cahier des charges :
\begin{itemize}
\item Le chargement et le déchargement du panier se fait manuellement en position basse.
\item La consigne de départ de cycle et l’information de fin de déchargement sont données
manuellement par l’opérateur.
\item Le chariot ne peut se déplacer que lorsque le panier est en position haute. Un voyant doit
s’allumer lorsque le chariot se déplace.
\item Les pièces doivent rester 30 secondes dans le banc de dégraissage.
\item Le cycle ne peut démarrer que si le chariot est à gauche et le panier en position basse.
\item Lorsque l’opération de déchargement est terminée, le chariot revient en position initiale.
\end{itemize}

\begin{center}
\includegraphics[width=.5\textwidth]{images/fig_03}
\end{center}

Données : 

\begin{center}
\includegraphics[width=.5\textwidth]{images/fig_04}
\end{center}


\subparagraph{}
\textit{A partir des tableaux ci-dessus, identifier les entrées sorties du poste de dégraissage.}

\subparagraph{}
\textit{Continuer le diagramme d’états du Document Réponse en utilisant les spécificités
technologiques de ce système en proposant une description sans puis avec états
composites.}

\subparagraph{}
\textit{Pour tous les états, préciser comment sont activées ou désactivées les sorties, selon le
mode de représentation choisi (sans et avec état composite).}


On souhaite améliorer la partie commande en ajoutant une fonctionnalité. Si l’opérateur donne
comme consigne « départ cycle sans trempage dcyst » au lieu de « départ cycle dcy », les pièces
doivent être envoyées directement au poste de déchargement sans passer par le poste de
dégraissage.

\subparagraph{}
\textit{Rajouter dans le diagramme d’état, les directives supplémentaires du cahier des charges
décrites ci-dessus.}


\begin{center}
\includegraphics[width=.95\textwidth]{images/fig_05}
\end{center}

\newpage

\section*{Unité de transfert de mise en pot}
\begin{center}
\includegraphics[width=.5\textwidth]{images/fig_06}
\end{center}
\setcounter{subparagraph}{0}

On s'intéresse à une unité de transfert rotatif destinée à la mise en pot alimentaire dont on
donne une description matérielle ainsi qu’un extrait du cahier des charges fonctionnel.

\begin{center}
\includegraphics[width=.5\textwidth]{images/fig_07}
\end{center}

Extrait du cahier des charges : 
\begin{itemize}
\item Les pots à remplir arrivent du poste de lavage par le tapis d’amenage.
\item Si le mode automatique est enclenché, les 3 taches (remplir, boucher et imprimer) doivent
être effectuées simultanément.
\item Dès que les 3 taches sont terminées, les pots doivent être transférés d’un poste à l’autre
(remplissage → bouchage → impression) par un disque de distribution motorisé par un
motoréducteur.
\item La position du disque doit être détectée par une cellule photo électrique. Le disque est en
position lorsqu’un trou est en face du détecteur.
\item La position du disque doit être maintenue par un vérin pneumatique indexeur.
\end{itemize}

\begin{center}
\includegraphics[width=.5\textwidth]{images/fig_08}
\end{center}

Entrée/Sorties du système :

\begin{center}
\includegraphics[width=.5\textwidth]{images/fig_09}
\end{center}

\subparagraph{}
\textit{Établir le diagramme d’état de cette unité de transfert de mise en pot à partir des
données du C.d.C.F.}

\subparagraph{}
\textit{Ajouter sur le diagramme d’états la gestion d’un arrêt d’urgence : lorsque l’arrêt d’urgence
(ARU) est enclenché, tous les états sont désactivés ; lors du réarmement (fin de l’arrêt
d’urgence), le cycle reprendra là où il s’était arrêté.}

\newpage 

\section*{Poste de compression de cartouches de chasse}
\begin{flushright}
\textit{D'après Patrick Beynet et Al., Les Sciences Industrielles de l'Ingénieur en PCSI -- MPSI, Éditions Ellipses.}
\end{flushright}
\begin{center}
\includegraphics[width=.5\textwidth]{images/fig_10}
\end{center}
\setcounter{subparagraph}{0}

\subsection*{Mise en situation et description du cycle de fonctionnement}

On s’intéresse à un cycle de fonctionnement à deux vérins caractérisant le comportement d’un
poste de compression de cartouches de chasse. Le système étudié doit permettre par
l’intermédiaire de deux vérins, les opérations de maintien et de compression d’une cartouche de
chasse.

L’opération de compression consiste à enfoncer une bourre dans le fond de la cartouche audessus
de la poudre. Les bourres d’une part et les cartouches d’autre part sont amenées par un
système de transfert dont cette étude ne fait pas l’objet.

Cycle de fonctionnement :
Dès que le départ cycle est donné, le vérin de maintien vient plaquer la cartouche contre un
appui afin de la maintenir. Le vérin de compression enfonce alors les bourres dans le fond de
l’étui de la cartouche. Il se retire et simultanément, le vérin de maintien libère la cartouche afin
qu’elle soit évacuée.

\begin{center}
\includegraphics[width=.5\textwidth]{images/fig_11}
\end{center}

On utilise (voir figure ci-dessus) :
\begin{itemize}
\item un bouton poussoir de départ cycle « dcy »;
\item un vérin simple effet comme vérin de maintien (pré-actionneur monostable : ordre de sortie
« VM »), et détecteurs fin de course « $vm_0$ » (tige rentrée) et « $vm_1$ » (tige sortie) ;
\item un vérin simple effet comme vérin de compression (préactionneur monostable : ordre de
sortie « VC »), et détecteurs fin de course « $vc_0$ » (tige rentrée) et « $vc_1$ » (tige sortie).
\end{itemize}

\subsection*{Travail demandé}
\subparagraph{}
\textit{Compléter ci-dessous le chronogramme décrivant un cycle de fonctionnement normal. Le
système est-il séquentiel ou combinatoire ?}


\begin{center}
\includegraphics[width=.3\textwidth]{images/fig_12}
\end{center}

\subparagraph{}
\textit{Recopier et compléter le diagramme d’états du système ci-dessous.}


\begin{center}
\includegraphics[width=.75\textwidth]{images/fig_13}
\end{center}

\subparagraph{}
\textit{Proposer un diagramme d’états de l’état composite « maintien et compression ». On pourra utiliser deux états disjoints pour lesquels on précisera l’activité.}

\section*{Château d'eau}
\begin{center}
\includegraphics[width=.5\textwidth]{images/fig_14}
\end{center}
\setcounter{subparagraph}{0}

Un château d’eau est alimenté par trois pompes P1, P2 et P3 en fonction de l’état des trois
détecteurs de niveau h1, h2 et h3. Un détecteur de niveau est à l’état 1 s’il est noyé.

\begin{center}
\includegraphics[width=.5\textwidth]{images/fig_15}
\end{center}


Cahier des charges 1 :
Un interrupteur m permet de mettre en fonctionnement l’installation.
La pompe Pi est en fonctionnement si l’interrupteur m est actionné et si le détecteur de niveau hi
n’est pas noyé.

\subparagraph{}
\textit{Établir le diagramme d’état décrivant le fonctionnement de ce premier cahier des
charges.}

Cahier des charges n°2 :

Le fonctionnement, décrit précédemment, fait apparaître une utilisation excessive de la pompe
P3 ce qui provoque son échauffement et diminue sa durée de vie. Pour éviter ces inconvénients,
on décide d’effectuer une permutation circulaire de l’utilisation des pompes, à chaque front
montant de h3 ou de l’interrupteur m.

\subparagraph{}
\textit{Établir le diagramme d’état décrivant le fonctionnement de ce deuxième cahier des
charges.}


\end{document}


