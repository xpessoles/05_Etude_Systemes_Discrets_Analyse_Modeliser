\documentclass[10pt]{article}
\input{style/coursHeadings}
\input{style/programHeadings}
\input{style/macros_SII}
\input{style/macros_Titres}
\input{style/macros_Frames}

%Si le boolen xp est vrai : compilation pour xabi
%Sinon compilation Damien
\newboolean{xp}
\setboolean{xp}{true}

\newboolean{prof}
\setboolean{prof}{true}

\newboolean{td}
\setboolean{td}{true}


\usepackage[%
    pdftitle={CI5 SED - Systèmes Logiques},
    pdfauthor={Xavier Pessoles},
    colorlinks=true,
    linkcolor=blue,
    citecolor=magenta]{hyperref}


\def\discipline{Sciences Industrielles de l'Ingénieur}
\def\xxtitre{\ifthenelse{\boolean{xp}}{
CI 5 : Étude du comportement des systèmes numériques}{
Chapitre  -- }}

\def\xxsoustitre{\ifthenelse{\boolean{xp}}{
Chapitre 1 -- Étude des systèmes logiques (appelés aussi combinatoires)}{
Partie  -- }}

\def\xxauteur{\ifthenelse{\boolean{xp}}{
Xavier \textsc{Pessoles} \\ 2013 -- 2014}{
}}

\def\xxpied{\ifthenelse{\boolean{xp}}{
CI 5 : Étude du comportement des systèmes numériques -- Applications\\
Chapitre 1 -- Étude des systèmes logiques}{
\xxtitre}}

\def\xxcathegorie{\ifthenelse{\boolean{xp}}{
2013 -- 2014 \\
Xavier \textsc{Pessoles}}{}}





%---------------------------------------------------------------------------


\begin{document}

\ifthenelse{\boolean{xp}}{\input{style/enteteXP}}{\input{style/enteteDI}}

\begin{center}
\large{\textsc{Applications directes}}
\end{center}

\vspace{.5cm}
\begin{flushright}
\textit{D'après ressources de Jean-Pierre Pupier.}
\end{flushright}

\subsection*{Algèbre de Boole}
\setcounter{subparagraph}{0}
\subparagraph{}
\textit{Simplifier les équations suivantes en utilisant uniquement l'algèbre de Boole :}

\begin{minipage}[c]{.47\linewidth}
$$S_1 = a+ab+abc$$
$$S_2 = \overline{a}bc+ac+(a+b)\overline{c}$$
$$S_3 = bc + ac + ab + b$$
\end{minipage}\hfill
\begin{minipage}[c]{.47\linewidth}
$$S_4 = a\overline{b}\overline{c}+\overline{a}\overline{c}+(a+b+c)\overline{c}$$
$$S_5 = (\overline{a}b+ab+a\overline{b})(c\overline{d}+\overline{c}\overline{d})+\overline{c}d(\overline{a}b+ab)$$
\end{minipage}


\subsection*{Logigramme}
\setcounter{subparagraph}{0}
\subparagraph{}
\textit{Simplifier l'équation suivante en utilisant uniquement l'algèbre de Boole puis tracer son logigramme :}
$$
F = b\overline{c}\overline{d}+ab\overline{d}+\overline{a}bc\overline{d}
$$

\subsection*{Allumez la lumière !}
\setcounter{subparagraph}{0}

Trois interrupteurs $a$, $b$, $c$ commandent l’allumage de deux lampes $R$ et $S$ suivant les conditions suivantes :
\begin{itemize}
\item dès qu’un ou plusieurs interrupteurs sont activés la lampe $R$ doit s’allumer;
\item la lampe $S$ ne doit s’allumer que si au moins deux interrupteurs sont activés.
\end{itemize}

\subparagraph{}
\textit{Calculer les expressions des fonctions binaires $R$ et $S$ et dessiner le logigramme.}


\subsection*{Étude d'un transcodeur}
\setcounter{subparagraph}{0}
\begin{minipage}[c]{.62\linewidth}
Considérons le système logique à 4 entrées $x_1$, $x_2$, $x_3$ et $x_4$ et 4 sorties $z_1$, $z_2$, $z_3$ et $z_4$ qui reçoit sur ses entrées le code binaire réfléchi d’un chiffre décimal et produit en sorties le code à excès de trois correspondant. Le code à excès de 3 d’un chiffre décimal $A$ est égal au code binaire naturel du nombre $A+3$. Un tel système est appelé transcodeur. La table de vérité suivante définit les 4 fonctions logiques réalisées par ce système. 
\subparagraph{}
\textit{Écrire les expressions minimales de chacune des 4 fonctions réalisées par le transcodeur.}% Il faudra tenir compte des combinaisons non utilisées.}
\subparagraph{}
\textit{Faire le logigramme correspondant aux 4 fonctions ainsi déterminées.}
\end{minipage}\hfill
\begin{minipage}[c]{.35\linewidth}
\begin{center}
\begin{tabular}{|c|cccc|cccc|}
\hline
 & $x_4$ & $x_3$ & $x_2$ & $x_1$ & $z_4$ & $z_3$ & $z_2$ & $z_1$ \\
\hline
\hline
0 & 0 & 0 & 0 & 0 & 0 & 0 & 1 & 1 \\
1 & 0 & 0 & 0 & 1 & 0 & 1 & 0 & 0 \\
2 & 0 & 0 & 1 & 1 & 0 & 1 & 0 & 1 \\
3 & 0 & 0 & 1 & 0 & 0 & 1 & 1 & 0 \\
4 & 0 & 1 & 1 & 0 & 0 & 1 & 1 & 1 \\
5 & 0 & 1 & 1 & 1 & 1 & 0 & 0 & 0 \\
6 & 0 & 1 & 0 & 1 & 1 & 0 & 0 & 1 \\
7 & 0 & 1 & 0 & 0 & 1 & 0 & 1 & 0 \\
8 & 1 & 1 & 0 & 0 & 1 & 0 & 1 & 1 \\
9 & 1 & 1 & 0 & 1 & 1 & 1 & 0 & 0 \\
\hline
\end{tabular}
\end{center}
\end{minipage}

\subsection*{Logigramme}
\setcounter{subparagraph}{0}

\begin{minipage}[c]{.5\linewidth}
\subparagraph{}
\textit{Donner l’équation de sortie $H$ : cette équation sera telle qu’aucun de ses termes ne soit complémenté.}
\end{minipage}\hfill
\begin{minipage}[c]{.45\linewidth}
\begin{center}
\includegraphics[width=.9\textwidth]{images/logigramme}
\end{center}
\end{minipage}


\subsection*{Pont}
\setcounter{subparagraph}{0}

Un pont peut soutenir 7 tonnes au maximum et on doit surveiller le poids des véhicules se présentant aux deux extrémités A et B où deux bascules mesurent les poids respectifs a et b des véhicules.

On suppose que tous les véhicules ont un poids inférieur à 7 tonnes :
\begin{itemize}
\item si un seul véhicule se présente la barrière correspondante A (ou B) s’ouvre;
\item si $a+b < 7$ tonnes, les barrières A et B s’ouvrent;
\item si $a+b > 7$ tonnes, la barrière correspondant au véhicule le plus léger s’ouvre;
\item si $a=b$, la barrière A s’ouvre en priorité.
\end{itemize}

$a$ et $b$ ne sont pas des variables binaires. Il convient donc de créer deux variables binaires X et Y et de reformuler l’énoncé du problème. 

\subparagraph{}
\textit{Déterminer les équations de fonctionnement de A et B en fonction de X et Y.}

\subparagraph{}
\textit{Tracer le logigramme.}


\subsection*{Perceuse}
\setcounter{subparagraph}{0}
Une perceuse est actionnée par un moteur électrique M. Le moteur ne peut fonctionner que si l’interrupteur de commande s est actionné et si les conditions de sécurité suivantes sont respectées :
\begin{itemize}
\item la protection de sécurité $p$ est en place;
\item le courant de surcharge $c$ n’est pas dépassé.
\end{itemize}

Outre ces conditions normales de fonctionnement une clé $k$ doit permettre de faire tourner le moteur sans que la protection soit en place.

\subparagraph{}
\textit{Établir l’équation logique permettant de commander le moteur M.}

\subparagraph{}
\textit{Faire le schéma électrique correspondant.}

\subsection*{Usine de brique}
\setcounter{subparagraph}{0}
Dans une usine de brique, on effectue un contrôle de qualité selon quatre 
critères : poids $P$, longueur $Lo$, largeur $la$, hauteur $H$. 1 correspond à une valeur correcte, 0 à une valeur incorrecte. Cela permet de classer les briques en 3 catégories :
\begin{itemize}
\item qualité A : le poids P et deux dimensions au moins sont corrects;
\item qualité B : le poids P seul est incorrect (les autres dimensions le sont, correctes) ou le poids étant correct deux dimensions au moins sont incorrectes;
\item qualité C : le poids P est incorrect ainsi qu’une ou plusieurs dimensions.
\end{itemize}

\subparagraph{}
\textit{Faire les tables de vérité et écrivez les équations des fonctions $A$, $B$ et $C$.}

\subparagraph{}
\textit{Simplifier ces équations.}


\subparagraph{}
\textit{Dessiner le logigramme à l’aide de 2 circuits intégrés contenant 3 ET-NON à trois entrées et de 1 circuit intégré contenant quatre OU-NON à deux entrées. On dispose des variables P, Lo, la, H sous une forme directe seulement.}


\subsection*{Usine de brique}
\setcounter{subparagraph}{0}
Le schéma logique ci-dessous est un additionneur soustracteur dont $S$ est la sortie, $R+$ est le report, $R-$ est la retenue, $r$ est le report de la retenue de poids inférieur. 
	
\subparagraph{}
\textit{Il est demandé de démontrer cette affirmation au regard des sorties S, R+, R-.}
\begin{center}
\includegraphics[width=.8\textwidth]{images/additionneur}
\end{center}

\end{document}
 

 
13°- Exercice - laveur de poudre
 
 

• La poudre est chargée et déchargée manuellement. Le fonctionnement est défini par le grafcet suivant et la réalisation par une carte numérique dont le principe est donné également en suivant.
  


 Donner sous forme de tableau le contenu de la mémoire EPROM(16x12).

 
14°- Exercices - Chargement de wagonnets [Extraits SI en CPGE FOUCHER]
• Dans une entreprise produisant des matières dangereuses, l'expédition obéit à une procédure garantissant la sécurité des personnes. Ces matières dangereuses sont chargées dans des wagonnets, ceux-ci sont tirés vers la sortie pour être expédiés au moyen d'une chaîne entraînée en translation par un pignon et un moteur (M3). Pour gagner la sortie, le wagonnet doit d'abord passer une porte de sécurité.
Le système possède trois moteurs (Ml, M2, M3) et quatre capteurs à contacts (a, b, c, d).
 
• Déroulement de la procédure de convoyage
o	chargement d’un wagonnet
o	accrochage du wagonnet chargé sur la chaîne qui l’entraîne vers la sortie (M3 fonctionne)
o	arrivée du wagonnet au niveau du capteur a
o	M3 s’arrête ainsi que le wagonnet
o	ouverture de la porte par M1
o	la porte ouverte active le capteur b qui envoie à M3 un signal de remise en marche ; le wagonnet franchit la porte et a est désactivé
o	l’arrivée du wagonnet au niveau du capteur d déclenche la fermeture par M2 de la porte restée ouverte
o	la fermeture de la porte active le capteur c qui déclenche le départ du wagonnet du poste d’expédition sans que celle-ci ne s’arrête.
o	Un autre wagonnet peut alors être accroché à la chaîne.

 Établir la table de vérité, les tableaux de Karnaugh et le logigramme décrivant le fonctionnent de cette installation.



\end{document}