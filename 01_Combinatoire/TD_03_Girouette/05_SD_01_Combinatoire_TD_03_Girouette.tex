\documentclass[10pt]{article}
\input{style/coursHeadings}
\input{style/programHeadings}
\input{style/macros_SII}
\input{style/macros_Titres}
\input{style/macros_Frames}

%Si le boolen xp est vrai : compilation pour xabi
%Sinon compilation Damien

\newif\ifprof
%\proftrue
\proffalse

\newif\ifxp
\xptrue
%\xpfalse

\newif\iftd
\tdtrue
%\tdfalse


\usepackage[%
    pdftitle={},
    pdfauthor={Xavier Pessoles},
    colorlinks=true,
    linkcolor=blue,
    citecolor=magenta]{hyperref}


\def\discipline{Sciences Industrielles de l'Ingénieur}
\def\xxtitre{%
\ifxp
CI ** : 
\else
\fi
}

\def\xxsoustitre{%
\ifxp
Chapitre ** -- 
\else
\fi}

\def\xxauteur{%
\ifxp
Xavier \textsc{Pessoles}
\else
\fi}

\def\xxpied{%
\ifxp
CI ** : **\\
Ch. ** : ** -- Cours TD
\else
\fi}



%---------------------------------------------------------------------------


\begin{document}
\ifxp
\input{style/enteteXP}
\else
\input{style/enteteDI}
\fi


\begin{warn}
Je compile avec pdfLaTeX.
\end{warn}
\setlength{\parskip}{0ex plus 0.2ex minus 0ex}
 \renewcommand{\contentsname}{}
 \renewcommand{\baselinestretch}{1}

\tableofcontents

 \renewcommand{\baselinestretch}{1.2}
\setlength{\parskip}{2ex plus 0.5ex minus 0.2ex}



\section{Titre 1}

\subsection{Sous titre 1}

\begin{minipage}[c]{.48\linewidth}
Le diagramme ci-contre recense les entrées et les sorties du système étudié :
\begin{itemize}
\item la variable vent ($V$) est à 1 lorsque le vent est trop fort;
\item la variable luminosité ($L$) est à 1 lorsque la luminosité devient importante;
\item la variable pluie ($P$) est à 1 lorsque le taux d’humidité devient important;
\item la variable Montée ($M$) est à 1 lorsqu’on désire commander la fermeture du store;
\item la variable descente (D$)$ est à 1 lorsqu’on désire commander l’ouverture du store. 
\end{itemize}
\end{minipage} \hfill
\begin{minipage}[c]{.48\linewidth}
\begin{center}
\includegraphics[width=.95\textwidth]{images/ES}
\end{center}
\end{minipage} 
Le store est toujours remonté ($M=1$), sauf en présence de soleil, sans pluie ni vent. Un système mécanique intégré au store coupe le moteur quand le store est complètement remonté bien que la commande soit maintenue.

Les entrées $V$, $L$ et $P$ seront simulées respectivement par 3 entrées ($SW_1$, $SW_2$ et $SW_3$) et les sorties $M$ et $D$ seront représentées par 2 leds ($LED_1$, $LED_2$). Toutes ces désignations sont relatives à la carte d’essai à base d’ispLSI1016 de chez Lattice.


\end{document}


