\documentclass[10pt]{article}
\input{style/coursHeadings}
\input{style/programHeadings}
\input{style/macros_SII}
\input{style/macros_Titres}
\input{style/macros_Frames}

%Si le boolen xp est vrai : compilation pour xabi
%Sinon compilation Damien

\newif\ifprof
%\proftrue
\proffalse

\newif\ifxp
\xptrue
%\xpfalse

\newif\iftd
\tdtrue
%\tdfalse


\usepackage[%
    pdftitle={},
    pdfauthor={Xavier Pessoles},
    colorlinks=true,
    linkcolor=blue,
    citecolor=magenta]{hyperref}


\def\discipline{Sciences Industrielles de l'Ingénieur}
\def\xxtitre{%
\ifxp
5 : Analyse et modélisation du comportement des systèmes 
\else
\fi
}

\def\xxsoustitre{%
\ifxp
Chapitre 1 -- Étude des systèmes combinatoires 
\else
\fi}

\def\xxauteur{%
\ifxp
Xavier \textsc{Pessoles}
\else
\fi}

\def\xxpied{%
\ifxp
5 : Analyse et modélisation du comportement des systèmes 
numériques\\
Ch.1 : Étude des systèmes combinatoires -- TD
\else
\fi}



%---------------------------------------------------------------------------


\begin{document}

\input{style/enteteXP}




\section{Store Somfy}
 
\begin{minipage}[c]{.48\linewidth}
On donne le diagramme des cas d’utilisation et le diagramme des exigences d’un store automatique.
\begin{center}
\includegraphics[width=.95\textwidth]{images/uc}
\end{center}
\end{minipage} \hfill
\begin{minipage}[c]{.45\linewidth}
\begin{center}
\includegraphics[width=.85\textwidth]{images/req}
\end{center}
\end{minipage} 


\subsection{Commande automatique du store}

\begin{minipage}[c]{.48\linewidth}
Le diagramme ci-contre recense les entrées et les sorties du système étudié :
\begin{itemize}
\item la variable vent ($V$) est à 1 lorsque le vent est trop fort;
\item la variable luminosité ($L$) est à 1 lorsque la luminosité devient importante;
\item la variable pluie ($P$) est à 1 lorsque le taux d’humidité devient important;
\item la variable Montée ($M$) est à 1 lorsqu’on désire commander la fermeture du store;
\item la variable descente (D$)$ est à 1 lorsqu’on désire commander l’ouverture du store. 
\end{itemize}
\end{minipage} \hfill
\begin{minipage}[c]{.48\linewidth}
\begin{center}
\includegraphics[width=.95\textwidth]{images/ES}
\end{center}
\end{minipage} 
Le store est toujours remonté ($M=1$), sauf en présence de soleil, sans pluie ni vent. Un système mécanique intégré au store coupe le moteur quand le store est complètement remonté bien que la commande soit maintenue.

Les entrées $V$, $L$ et $P$ seront simulées respectivement par 3 entrées ($SW_1$, $SW_2$ et $SW_3$) et les sorties $M$ et $D$ seront représentées par 2 leds ($LED_1$, $LED_2$). Toutes ces désignations sont relatives à la carte d’essai à base d’ispLSI1016 de chez Lattice.


\subparagraph{}
\textit{Réaliser la table de vérité correspondant au fonctionnement du store.}

\subparagraph{}
\textit{Déterminer les équations logiques pour M et pour D. }

\subparagraph{}
\textit{Simplifier éventuellement vos équations.}

\subparagraph{}
\textit{Réaliser le logigramme en utilisant les portes de votre choix.}

\subsection{Affichage de la direction du vent}

\begin{minipage}[c]{.7\linewidth}
Il est possible que l'utilisateur ait à sa disposition une girouette pour connaitre la direction du vent. 
La direction du vent est décomposée en 8 secteurs de 45° chacun (rose des vents). Chaque secteur est désigné par une combinaison des quatre points cardinaux Nord (N), Est (E), Sud (S) et Ouest (O).

L'information est traitée de la façon suivante : 
\begin{center}
\includegraphics[width=.95\textwidth]{images/chaineMesure}
\end{center}

\end{minipage} \hfill
\begin{minipage}[c]{.28\linewidth}
\begin{center}
\includegraphics[width=.95\textwidth]{images/RoseVents}
\end{center}
\end{minipage} 



La détection de la direction du vent est assurée par une girouette. Celle-ci fournit une information logique codée en binaire réfléchi sur 4 bits représentative de la direction du vent (Y4, Y3, Y2 et Y1, Y1 étant le LSB).

Le premier transcodage permet la traduction de cette information en binaire pur (D4, D3, D2 et D1).

Le second transcodage traduit le mot écrit en binaire pur en une information pouvant être lue sur des afficheurs à 7 segments (7 segments de a à g et deux bits pour sélectionner l’afficheur).

L'affichage de la direction du vent (secteur) est visible sur 2 afficheurs. 

\begin{obj}

Les objectifs sont :
\begin{itemize}
\item d'établir la table de vérité du transcodage binaire réfléchi en binaire pur afin d’en déduire les équations logiques. Ces équations seront implémentées dans le circuit programmable.
\item d’implanter dans un circuit logique programmable la fonction transcodage pour afficher la direction du vent sur deux afficheurs 7 segments directement à partir du code binaire réfléchi.
\end{itemize}

\end{obj}

\subparagraph{}
\textit{Réaliser la table de vérité permettant de traduire le code binaire réfléchi en code binaire pur.}

\subparagraph{}
\textit{Quel est l’intérêt du binaire réfléchi dans le cas de la girouette ? Quel type de capteur est –il possible de trouver dans la girouette ?}

\subparagraph{}
\textit{Donner les équations de D4, D3, D2 et D1.}

\subparagraph{}
\textit{Tracer le logigramme afin de réaliser les signaux précédents. }


\subsection{Étude du second transcodage : affichage de la direction du vent}


\begin{minipage}[c]{.5\linewidth}

Elle traduit le code issu de la girouette (représentatif de la direction du vent) en un code dit 7 segments qui permettra d'afficher la direction du vent. On conservera en entrée le code binaire réfléchi.
Un afficheur 7 segments est constitué de 7 segments lumineux (leds). Chacun des segments est repéré par une lettre, de a jusqu'à g comme indiqué ci-contre. 

On s’intéresse à l’affichage de la direction Nord ou Sud en fonction de la position du codeur absolu.

\subparagraph{}
\textit{Exprimer l’équation logique des segments de l’afficheur en fonction des entrées Yi.}

\end{minipage} \hfill
\begin{minipage}[c]{.15\linewidth}
\begin{center}
\includegraphics[width=.95\textwidth]{images/7segments}
\end{center}
\end{minipage} \hfill
\begin{minipage}[c]{.28\linewidth}

\begin{center}
\includegraphics[width=.9\textwidth]{images/Orientations}
\end{center}
\end{minipage} 



\subsection*{Documentation du codeur}

\begin{minipage}[c]{.7\linewidth}
C’est un codeur numérique 110E Vishay. Ce fabricant propose une gamme de plusieurs encodeurs (de 2 à 4bits) fournissant une information absolue ou relative.

\begin{enumerate}
\item Les cases noircies représentent le « 1 » logique.
\item La colonne « G » représentent le point commun, relié ici au +5V.
\item Des résistances de tirage imposent du « 0 » logique en cas de non connexion au +5V via la borne « G ».
\item Le capteur est branché sur la carte via le connecteur J8.
\end{enumerate}
La colonne 1 représente le bit de poids faible (LSB) et la dernière colonne le but de poids faible (MSB). 

\begin{center}
\includegraphics[width=.95\textwidth]{images/codeur}
\end{center}
\end{minipage} \hfill
\begin{minipage}[c]{.28\linewidth}

\begin{center}
\includegraphics[width=.9\textwidth]{images/codeur_table}
\end{center}
\end{minipage} 

\end{document}


